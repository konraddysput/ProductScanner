\documentclass[12pt,a4paper,leqno,oneside,titlepage]{mwrep}
\usepackage{polski}
\usepackage[utf8]{inputenc}
\usepackage[T1]{fontenc}
\usepackage{graphicx}
\usepackage{tabularx,ragged2e,booktabs,caption}
\usepackage{listings}
\usepackage[dvipsnames]{xcolor}
\usepackage{import}
\usepackage{grffile}
\usepackage{longtable}
\usepackage{ucs}
\usepackage{blindtext}
\usepackage{wrapfig}
\usepackage{enumitem}
\usepackage{subfig}


\setlength{\oddsidemargin}{1.4cm}
\setlength{\textwidth}{14cm}
\setlength{\topmargin}{0in}
\setlength{\textheight}{21.1cm}
\linespread{1.05}

%imported packages settings

%1. listings
\lstset{language=[Sharp]C,captionpos=b,tabsize=3,frame=lines,extendedchars=true,inputencoding=utf8x,
keywordstyle=\color{blue},commentstyle=\color{darkgreen},stringstyle=\color{red},
numbers=left,numberstyle=\tiny,numbersep=5pt,breaklines=true,showstringspaces=false,basicstyle=\footnotesize,emph={label},extendedchars=\true,
literate={ą}{{\k{a}}}1
{Ą}{{\k{A}}}1
{ę}{{\k{e}}}1
{Ę}{{\k{E}}}1
{ó}{{\'o}}1
{Ó}{{\'O}}1
{ś}{{\'s}}1
{Ś}{{\'S}}1
{ł}{{\l{}}}1
{Ł}{{\L{}}}1
{ż}{{\.z}}1
{Ż}{{\.Z}}1
{ź}{{\'z}}1
{Ź}{{\'Z}}1
{ć}{{\'c}}1
{Ć}{{\'C}}1
{ń}{{\'n}}1
{Ń}{{\'N}}1}


\begin{document}

\begin{titlepage}
\begin{center}
{\large POLITECHNIKA POZNAŃSKA\\ WYDZIAŁ ELEKTRYCZNY\\ INSTYTUT AUTOMATYKI, ROBOTYKI I INŻYNIERII INFORMATYCZNEJ\par}
\end{center}
\vspace{1.5cm plus 1fill}
\begin{center}
{\bf \Large Konrad Dysput\par}
\end{center}
\vspace{1.5cm plus 1mm minus 2mm}
\begin{center}
{\large PRACA DYPLOMOWA MAGISTERSKA\par}
\end{center}
\vspace{1.5cm plus 1mm minus 2mm}
\begin{center}
{\huge\textbf{Klasyfikacja produktów na podstawie analizy zdjęć półek sklepowych }\par}
\vspace{1.5cm plus 1.5fill}
\begin{flushright}
{\large Promotor: dr inż. Jarosław Bąk}
\end{flushright}
\vspace{4cm plus .1fill}
{\large Poznań,\space 2018\par}
\end{center}
\end{titlepage}
\newpage
\begin{center}
{\large\textbf{Streszczenie}}
\end{center}
	W pracy przedstawiono koncepcję realizacji systemu analizy produktów na półce sklepowej. Opisano w niej metody detekcji przedmiotów na podstawie analizy obrazu oraz wnioskowania przy użyciu technologii semantycznej. Przedstawiono sposób realizacji przykładowego systemu - ProductScanner. Poprzez wykorzystanie aplikacji mobilnej użytkownik może udostępnić zdjęcie oraz otrzymać wynik wnioskowania bazującego na przedmiotach znajdujących się na zdjęciu.
	\newline
\begin{center}
	{\large\textbf{Abstract}}
\end{center}
	In thesis it was showed concept of system that allows to analyse products on shop shelf. There were described methods of detection of objects based on image analysis and web semantic. There was showed realisation of sample system - ProductScanner. By using mobile application user can upload image and receive inference result based on objects on the picture.


\tableofcontents

\chapter{Wstęp}
\import{sections/}{Wstep.tex}

\chapter{Opis narzędzi}
\import{sections/}{Funkcjonalnosc.tex}

\chapter{Spis wymagań i założeń}
\import{sections/}{Projekt.tex}

\chapter{Architektura systemu}
\import{sections/}{Architektura.tex}

\chapter{Implementacja}
\import{sections/}{Implementacja.tex}

\chapter{Użytkowanie aplikacji}
\import{sections/}{Testy.tex}

\chapter{Zakończenie}
\import{sections/}{Zakonczenie.tex}

\import{sections/}{Bibliografia.tex}
\end{document}