\begin{thebibliography}{1}
%koncepcja
\bibitem{CNTK} Microsoft Cognitive Toolkit (CNTK)
\\
https://docs.microsoft.com/en-us/cognitive-toolkit/

\bibitem{VOTT} Visual Object Tagging Tool (VOTT)
\\
https://github.com/CatalystCode/VOTT

\bibitem{Tensorflow} Biblioteka Tensorflow
\\
https://www.tensorflow.org/

\bibitem{fasterrcnn} Shaoqing Ren, Kaiming He, Ross Girshick, Jian Sun,  {\em Faster R-CNN: Towards Real-Time Object Detection with Region Proposal Networks} , Konferencja Neural Information Processing Systems (NIPS), 2015

\bibitem{fastrcnn} Ross Girshick, Jeff Donahue, Trevor Darrell, Jitendra Malik, {\em Rich feature hierarchies for accurate object detection and semantic segmentation}, Konferencja IEEE dotyczcząca przetwarzania obrazu oraz rozpoznawania wzorów (Computer Vision and Pattern Recognition) (CVPR), 2014


\bibitem{Anaconda} System zarządzania bibliotekami - Anaconda
\\
https://conda.io/docs/index.html

\bibitem{Tensorflow} System zarządzania bibliotekami w środowisku uruchomieniowym Python
\\
https://pypi.org/project/pip/

%założenia

\bibitem{Docker} Docker
\\
https://www.docker.com/
%architektura
\bibitem{REST} Representational State Transfer
\\
https://www.ics.uci.edu/~taylor/documents/2002-REST-TOIT.pdf
\bibitem{AMQP}Advanced Message Queuing Protocol
\\
https://www.amqp.org/
\bibitem{RabbitMQ} RabbitMQ
\\
https://www.rabbitmq.com/
\bibitem{JSON} JSON
\\
https://www.json.org/
\bibitem{SignalR} Komunikacja w czasie rzeczywistym - SignalR
\\
https://docs.microsoft.com/en-US/aspnet/core/signalr/?view=aspnetcore-2.1
\bibitem{Ionic} Ionic
\\
https://ionicframework.com
\bibitem{Cordova} Cordova
\\
https://cordova.apache.org/

\bibitem{Angular} Angular 5
\\
https://angular.io/
\bibitem{CORS}CORS
\\
https://developer.mozilla.org/en-US/docs/Web/HTTP/CORS
\bibitem{Protege}Protege
\\
https://protege.stanford.edu/

\bibitem{SWRL}Semantic Web Rule Language (SWRL)
\\
https://www.w3.org/Submission/SWRL/
%implementacja

\bibitem{Postman}Narzędzie Postman
\\
https://www.getpostman.com/


\bibitem{Fiddler}Narzędzie Fiddler
\\
https://www.telerik.com/fiddler

\bibitem{Protege} Protege
\\
https://protege.stanford.edu/

\bibitem{Powershell} Powershell
\\
https://docs.microsoft.com/pl-pl/powershell/scripting/powershell-scripting

\end{thebibliography}

Dostępność adresów do stron internetowych została sprawdzona w dniu 30.09 2018r.