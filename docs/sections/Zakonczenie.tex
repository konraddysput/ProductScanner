Niniejsza praca pokazuje możliwość połączenia wielu odrębnych technologii w celu stworzenia systemu analizy i klasyfikacji zdjęć z półek sklepowych. Zaimplementowana przykładowa aplikacja zawiera wszystkie stawiane przed nią funkcjonalności. Użytkownik dzięki niej może przeanalizować wysłane zdjęcie z telefonu komórkowego. System zakłada możliwość integracji z~wieloma mikroserwisami dzięki komunikacji przy pomocy protokołu AMQP. Dzięki niej otrzymano możliwość integracji z wieloma środowiskami programistycznymi wykorzystanymi do użycia poszczególnych technologii. Stawiany problem dotyczący możliwości przetwarzania zdjęć oraz dalszego wnioskowania przy pomocy sieci semantycznej został uznany za ukończony. System zapewnia możliwość dalszych rozszerzeń o dodatkowe moduły lub obiekty klasyfikacji. 

W ramach pracy zostały zrealizowane badania dotyczące najlepszego sposobu przetwarzania obrazu w celu detekcji przedmiotów z półek sklepowych. Zostało stworzone porównanie dostępnych bibliotek oraz usług chmurowych. W~pracy wskazano wady oraz zalety technologii chmurowych oraz bibliotek CNTK i Tensorflow. 

Przeprowadzone testy aplikacji potwierdzają poprawność działania poszczególnych założonych funkcjonalności. Zostały one wykonane w środowisku programistycznym oraz docelowym produkcyjnym na maszynie wirtualnej w chmurze. Aplikacja mobilna została przetestowana na urządzeniu mobilnym - Samsung Galaxy S6. Stworzony kod zakłada możliwość integracji aplikacji z urządzeniami iPhone. Nie dokonano testów na urządzeniach marki Apple. Było to spowodowane brakiem iPhone oraz dużymi kosztami zakupu telefonu.

ProductScanner jest rozproszoną aplikacją internetową umożliwiającą przetwarzanie zdjęć w celu klasyfikacji produktów oraz udostępnia mechanizmy wnioskowania dla uzyskanych wyników. Dalsze prace projektowe zakładają możliwość rozszerzenia modelu klasyfikacji produktów o kolejne rodzaje artykułów. Ponadto zwiększenie liczby rozpoznawanych przedmiotów wymaga uzupełnienia ontologii o dodatkowe klasy i reguły SWRL.