Współczesne technologie umożliwiają w bardzo łatwy sposób zautomatyzowanie pewnych czynności dokonywanych przez użytkowników aplikacji. Poza przyspieszeniem pracy eliminują one możliwość popełnienia błędów. Ponadto ewentualne pomyłki lub niedopatrzenia zaimplementowane w aplikacji można łatwo naprawić, dzięki czemu problem może zostać wyeliminowany u wszystkich użytkowników jednocześnie. 

%Automatyzacja działań pracowników firm jest niezbędnym elementem każdej organizacji.
 Dużym wyzwaniem postawionym przed sieciami sklepów jest dostosowanie w każdym z nich odgórnie przyjętych zasad dotyczących ułożenia produktów. Pracownik poświęca bardzo dużo czasu na odpowiednie ułożenie towaru na półkach. Ponadto zmiany umów sponsorskich oraz dynamiczne działanie firmy może spowodować, że wiedzę osoby za to odpowiedzialnej należy na bieżąco aktualizować. Może to spowodować błędy w ustawieniu przedmiotów oraz kary, które producenci produktów mogą nanieść na sklep. Innym bardzo ważnym czynnikiem analizy produktów jest bezpieczeństwo. Żadna z firm nie chciałaby, aby środki chemiczne lub alkohole znalazły się w zasięgu ręki dziecka.

\section{Idea systemu}
Wiele korporacji w branży sprzedaży detalicznej posiada zbliżone aplikacje umożliwiające detekcję przedmiotów na półkach sklepowych. Jest to oprogramowanie pisane na życzenie zamawiającego, bazujące na danych z hurtowni danych korporacji. Liczba publicznych rozwiązań jest bardzo ograniczona. Idea stworzenia oprogramowania do detekcji i analizy produktów wywodzi się z ogromnych strat finansowych u sprzedawców, którzy nie zaznajomili się z wytycznymi firmy. Powoduje to sytuację, w której firma płacąca za lepsze położenie produktu na półce względem innych traci swój przywilej mimo podpisanych umów. Kary finansowe za niepoprawne położenie produktów nie są jedynym zagrożeniem dla sklepów. Kluczowym aspektem jest bezpieczeństwo oraz przyjęte normy firmy. Organizacja zmieniając je lub wprowadzając nowe zasady chciałaby, aby jak najszybciej zostały one dostosowane i sprawdzone. Stworzenie systemu, który umożliwi automatyczne wykrycie nieprawidłowego ułożenia produktu, rozwiązuje przedstawione wyżej 	problemy oraz zapewnia kontrolę nad wyposażeniem półek sklepowych. Jest to możliwe dzięki analizie obrazu oraz przetwarzaniu reguł semantycznych.

Niniejsza praca ma na celu zaprojektowanie systemu, który wyeliminuje problem pozycjonowania produktów na półkach sklepowych. Użytkownik przychodzący do sklepu, korzystając z aplikacji może w prosty sposób sprawdzić poprawne ułożenie produktów na półkach oraz relacje między nimi. W tym przypadku żmudny i długi proces może zostać ograniczony tylko do zrobienia kilku zdjęć. 

\section{Cel i zakres pracy}
Niniejsza praca opisuje metody detekcji przedmiotów znajdujących się na obrazie oraz rozszerzenie informacji dotyczących danego zdjęcia poprzez zastosowanie sieci semantycznej. Celem pracy jest stworzenie oprogramowania przedstawiającego możliwość wykorzystania wnioskowania semantycznego opartego o detekcję przedmiotów na podstawie analizy obrazu. Dzięki temu użytkownik aplikacji będzie mógł w krótkim czasie otrzymać informację o niepoprawnym położeniu produktów na półce sklepowej na podstawie przyjętych reguł. W tym celu stworzono system, który nazwano ProductScanner.

Użytkownik posiadający aplikację będzie mógł przesłać za pomocą dowolnego urządzenia mobilnego zdjęcie z aparatu lub galerii zdjęć. Dzięki niemu rozproszone aplikacje przeprowadzą operacje przetwarzania obrazu oraz wnioskowania na podstawie istniejącej ontologii. Czas przetwarzania może być zmienny i niekoniecznie krótki. W związku z tym aplikacja zakłada możliwość aktualizacji danych w aplikacji internetowej w czasie rzeczywistym. Użytkownik może dokonywać analizy zdjęcia zrobionego przed chwilą oraz przeglądać archiwum dostępnych zdjęć w programie. Wszystkie wymienione operacje użytkownik może wykonać przy użyciu telefonu komórkowego.

Przygotowanie systemu wymaga wiedzy dotyczącej sieci semantycznych, aplikacji rozproszonych oraz uczenia maszynowego. Aby umożliwić łatwą integrację dostępu do zdjęć oraz dostosować przygotowaną pracę do warunków rzeczywistych stworzono aplikację mobilną. Ponadto założono możliwość korzystania z aplikacji tylko dla zalogowanych użytkowników. Przedstawione założenia oraz plany mają na celu dostosowanie aplikacji do rzeczywistych potrzeb firm.

%Tutaj powinien się znalezć tekst opisujący po kolei każdy rozdział.