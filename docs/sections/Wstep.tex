Współczesne technologie umożliwiają w bardzo łatwy sposób zautomatyzowanie pewnych czynności dokonywanych przez użytkowników aplikacji. Poza przyspieszeniem pracy eliminują one możliwość popełnienia błędów. Ponadto ewentualne pomyłki lub niedopatrzenia zaimplementowane w aplikacji można w bardzo łatwy sposób naprawić, dzięki temu problem można wyeliminować ~u wszystkich użytkowników jednocześnie. 

Wykorzystanie nowych technologii, takich jak przetwarzanie zdjęć lub sieć semantycza, może być problematyczne pod względem użytych środowisk programistycznych. Liczba bibliotek oraz sposób ich użycia jest ograniczona lub we wczesnej fazie rozwoju. Powoduje to ogromne trudności z ich integracją. W tym celu pomocne może być zastosowanie rozproszonego środowiska, które zapewni warstwę abstrakcji nad interfejsem programistycznym poszczególnych środowisk programistycznych.

\section{Idea systemu}
Wiele korporacji w branży retail posiada zbliżone aplikacje umożliwiające detekcję przedmiotów na półkach sklepowych. Jest to oprogramowanie pisane na życzenie zamawiającego, bazujące na danych z hurtowni danych korporacji. Liczba publicznych rozwiązań jest bardzo ograniczona. 
Idea stworzenia oprogramowania do detekcji i analizy produktów wywodzi się z ogromnych strat finansowych u vendorów sklepów, którzy nie zaznajomili się z wytycznymi firmy. Powoduje to sytuację, w której firma płacąca za lepsze położenie produktu na półce względem innych traci swój przywilej mimo podpisanych umów. 

Niniejsza praca ma na celu przedstawienie systemu, który wyeliminuje problem pozycjonowania produktów na półkach sklepowych. Użytkownik przychodzący do sklepu, korzystając z aplikacji może w prosty sposób sprawdzić poprawne ułożenie produktów na półkach oraz relacje między nimi. W tym przypadku żmudny i długi proces może zostać ograniczony tylko do stworzenia kilku zdjęć. 

\section{Cel i zakres pracy}
Niniejsza praca opisuje metody detekcji przedmiotów znajdujących się na obrazie oraz rozszerzenie informacji dotyczących danego zdjecia poprzez zastosowanie sieci semantycznej. Celem pracy jest stworzenie oprogramowania przedstawiającego możliwość integracji wspomnianych technologii.

Użytkownik posiadający aplikację, będzie mógł przesłać za pomocą dowolnego urządzenia mobilnego zdjęcie z aparatu lub galerii zdjęć. Dzięki niemu rozproszone aplikacje przeprowadzą operacje przetwarzania obrazu oraz wnioskowania na podstawie istniejącej ontologii. Czas przetwarzania może być zmienny i niekoniecznie krótki. W związku z tym aplikacja zakłada możliwość aktualizacji danych w aplikacji internetowej w czasie rzeczywistym. Użytkownik może zarówno dokonywać analizy zdjecia stworzonego przed chwilą oraz przeglądać archiwum dostępnych zdjęć w programie.

Przygotowanie systemu zakłada przyswojenie wiedzy dotyczącej sieci semantycznych, aplikacji rozproszonych oraz analizy obrazu. Aby umożliwić łatwą integrację zdjęć oraz dostosować przygotowaną pracę do warunków rzeczywistych, dodatkowo stworzono aplikację mobilną. Ponadto założono możliwość korzystania z aplikacji tylko dla zalogowanych użytkowników. Przedstawione założenia oraz plany mają na celu dostosowanie aplikacji do rzeczywistych potrzeb firm.