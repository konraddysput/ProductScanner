Celem rozdziału jest przedstawienie istniejących rozwiązań dostępnych w chmurze oraz bibliotek programistycznych. Wybranie odpowiedniego rozwiazania służącego analizie obrazu było jednym najważniejszym elementem badań pracy magisterskiej. Zmiana wyboru przetwarzania obrazu mogła powodować dokonywanie inwazyjnych zmian w strukturze projektu jak i sposobie komunikacji poszczególnych jego części. 

\section{Przetwarzanie obrazu w chmurze Azure} {
Usługi chmurowe proponowane przez największych dostawców takich jak Amazon Web Services, Microsoft Azure, lub Google Cloud, dostarczają zaawansowane rozwiązania z zakresu przetwarzania obrazu oraz uczenia maszynowego. Ich wybór wymaga udostępniania zdjęć koniecznych do przygotowania modelu uczenia maszynowego do serwera w chmurze oraz dokładną analizę udostępnionego interfejsu programistycznego usługodawcy.Stworzony model na podstawie obrazów musiałby być skrupulatnie przeanalizowany, ponieważ koszt przechowywania zdjęć jak i czas związany z ich analizą może spowodować zwiększenie kosztów przetwarzania danych. Jednym z celów architektów rozwiązań w chmurowych jest jak największe rozbicie modelu obliczeniowego. W przypadku projektowanego rozwiązania, rosnącej ilości obrazów oraz analizowanych elementów mogłaby spowodować zbytnią złożoność systemu. 

Dokładną analizę dostępnych usług w chmurze dokonano przy użyciu Microsoft Azure. Na jej korzyść przemawiało bardzo duża liczba dostępnych serwisów służących przetwarzaniu obrazów oraz koszta ich wykorzystania. Sposób tworzenia modelu poddawanego analizie opierał się na przesłaniu do serwera zdjęć oraz zaznaczeniu poszukiwanego elementu na obrazie. Wykorzystanie chmury Azure uniemożliwiłoby przejście na inne rozwiazania w przypadku napotkania problemów z wykorzystaniem zasobów oraz wykorzystanie interfejsu programistycznego, który mógłby być nie kompatybilny z innymi rozwiązaniami. Ryzyko wykorzystania chmury mogłoby spowodować niepowodzenie w dalszej częsci projektu. 
}
\section{Zastosowanie biblioteki CNTK}{
Firma Microsoft przygotowała otwartą bibliotekę CNTK \cite{CNTK}, która służy do rozwiązywania problemów uczenia maszynowego przy wykorzystaniu głębokich sieci neuronowych. W tym celu do zuczenia wykorzystuje stochastyczny gradient oraz wsteczną propagację. Biblioteka umożliwia wykorzystaniu różnych języków programowania: Python, C\#, C++ oraz działa na 64-bitowych systemach operacyjnych Linux oraz Windows. W Internecie można znaleźć wiele przykładów wykorzystanie biblioteki lecz jej nawiększym minusem jest brak dobrej dokumentajci. Jej brak uniemożliwił dalszą analizę komponentów niezbędnych do stworzenia przykładowego rozwiązania. Biblioteka nie cieszy się dobrej opinią względem jej konkurencji - Tensorflow. Jest to spowodowane dalszą koniecznością rozwoju oraz wsparcia innych systemów operacyjnych. 
}

\section{Wykorzystanie biblioteki Tensorflow}{
Biblioteka Tensorflow \cite{Tensorflow} jest wykorzystywana do rozwiazywania problemów opartych o uczenie maszynowe lub głębokie sieci neuronowe. Szybkość działania oraz łatwość w dostępie do danych zawdzięcza dzięki językowi C++ oraz Python. Jej twórcy sukcesywnie zwiększają liczbę dostępnych interfejsów programistycznych dla różnych środowisk uruchomieniowych. Jej podstawowym oraz najbardziej rozbudowanym intefejsem programistycznym jest kompilacja udostępniona na programy wykonane przy użyciu języka python. Biblioteka działa w oparciu o wykorzystanie procesorów, procesorów wbudowanych, mikroprocesory lub karty graficzne, niezależnie od architektury maszyny na której została uruchomiona. Narzędzie jest wykorzystywane w wielu programach - miedzy innymi do analizy zdjęć przez firmę Google w chmurze. 

Wykorzystanie biblioteki na systemach operacyjnych Windows z jezykiem python, wiąże się z koniecznością dostosowywania dostępnych pakietów z wykorzystaniem systemu zarządzania bibliotekami Anaconda \cite{Anaconda}. Domyślne narzędzie dostarczone wraz ze środowiskiem uruchomieniowym PIP, generuje wiele problemów dla bibliotek, które są niezbędne do uruchomienia Tensorflow. Programista tworzący oprogramowanie na systemie Windows z wykorzystaniem tensorflow powinien pamiętać o wielu dodatkowych wytycznych niezbednych do uruchomienia skryptów. Do nich mogą należeć dodatkowe ścieżki środowiskowe (przykładowo: PYTHONPATH) lub konieczność uruchomienia wirtualnego środowiska narzędzia Anaconda. 

Interfejs programistyczny biblioteki Tensorflow zapewnił umożliwił stworzenie skryptów odpowiedzialnych za detekcje przedmiotów znajdujących się na obrazie. Dodatkowo umożliwia ona uzyskanie informacji o pozycji lub klasyfikacji obiektu. Dostarczona przez autorów dokumentacja umożliwiła sprawną naukę sposobu działania biblioteki oraz płynną implementację założeń projektowych. W stworzonym systemie biblioteka Tensorflow została użyta w celu detekcji przedmiotów znajdujących sie na obrazie.
}
