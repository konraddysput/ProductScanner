Celem rozdziału jest przedstawienie istniejących rozwiązań dostępnych w chmurze oraz bibliotek programistycznych. Wybranie odpowiedniego rozwiązania służącego analizie obrazu było najważniejszym elementem badań pracy magisterskiej. Zmiana wyboru przetwarzania obrazu mogła powodować dokonywanie inwazyjnych zmian w strukturze projektu jak i sposobie komunikacji poszczególnych jego części. 

\section{Przetwarzanie obrazu w chmurze Azure} {
Usługi chmurowe proponowane przez największych dostawców, takich jak Amazon Web Services, Microsoft Azure lub Google Cloud, dostarczają zaawansowane rozwiązania z zakresu przetwarzania obrazu oraz uczenia maszynowego. Ich wybór wymaga udostępniania zdjęć koniecznych do przygotowania modelu uczenia maszynowego do serwera w chmurze oraz dokładną analizę udostępnionego interfejsu programistycznego usługodawcy. Model stworzony na podstawie obrazów musiałby być skrupulatnie przeanalizowany, ponieważ koszt przechowywania zdjęć jak i czas związany z ich analizą może spowodować zwiększenie kosztów przetwarzania danych. Jednym z celów architektów rozwiązań chmurowych jest jak największe rozbicie modelu obliczeniowego. W przypadku projektowanego rozwiązania rosnąca liczba obrazów oraz analizowanych elementów mogłaby spowodować zbytnią złożoność systemu. 

Dokładną analizę dostępnych usług w chmurze dokonano przy użyciu Microsoft Azure. Na jej korzyść przemawiała bardzo duża liczba dostępnych serwisów służących przetwarzaniu obrazów oraz koszty ich wykorzystania. Sposób tworzenia modelu poddawanego analizie opierał się na przesłaniu do serwera zdjęć oraz zaznaczeniu poszukiwanego elementu na obrazie. Użycie chmury Azure spowodowałoby konieczność implementacji algorytmów z bibliotekami dostarczonymi przez usługodawcę. W przypadku braku dostępnych metod niezbędnych do rozwiazania problemu stawianego oprogramowaniu, wymmagałoby reimplementacji bardzo dużej ilości kodu. 

\section{Zastosowanie biblioteki CNTK}{
Firma Microsoft przygotowała otwartą bibliotekę CNTK \cite{CNTK}, która służy do rozwiązywania problemów uczenia maszynowego przy wykorzystaniu głębokich sieci neuronowych. W tym celu korzysta ze stochastycznego gradientu oraz wstecznej propagacji. Biblioteka umożliwia używanie różnych języków programowania: Python, C\#, C++ oraz działa na 64-bitowych systemach operacyjnych Linux oraz Windows. W Internecie można znaleźć wiele przykładów korzystania z biblioteki, lecz jej nawiększym minusem jest brak dobrej dokumentacji. Uniemożliwiło to dalszą analizę komponentów niezbędnych do stworzenia przykładowego rozwiązania. Biblioteka CNTK nie cieszy się dobrą opinią względem konkurencji - Tensorflow. Jest to spowodowane dalszą koniecznością rozwoju oraz wsparcia innych systemów operacyjnych. 
}

\section{Zastosowanie biblioteki Tensorflow}{
Biblioteka Tensorflow \cite{Tensorflow} jest wykorzystywana do rozwiązywania problemów opartych o uczenie maszynowe lub głębokie sieci neuronowe. Szybkość działania oraz łatwość w dostępie do danych zawdzięcza językowi C++ oraz Python. Jej twórcy sukcesywnie zwiększają liczbę dostępnych interfejsów programistycznych dla różnych środowisk uruchomieniowych. Jej podstawowym oraz najbardziej rozbudowanym interfejsem programistycznym jest kompilacja udostępniona na programy wykonane przy użyciu języka Python. Biblioteka działa w oparciu o wykorzystanie procesorów, procesorów wbudowanych, mikroprocesorów lub karty graficzne, niezależnie od architektury maszyny na której została uruchomiona. Narzędzie jest wykorzystywane w wielu programach - między innymi do analizy zdjęć przez firmę Google w chmurze. 

Wykorzystanie biblioteki na systemach operacyjnych Windows z językiem Python wiąże się z koniecznością dostosowywania dostępnych pakietów za pomocą systemu zarządzania bibliotekami Anaconda \cite{Anaconda}. Domyślne narzędzie dostarczone wraz ze środowiskiem uruchomieniowym PIP generuje wiele problemów dla bibliotek, które są niezbędne do uruchomienia Tensorflow. Programista tworzący oprogramowanie na systemie Windows z wykorzystaniem Tensorflow powinien pamiętać o wielu dodatkowych wytycznych niezbędnych do uruchomienia skryptów. Do nich mogą należeć dodatkowe ścieżki środowiskowe (przykładowo: PYTHONPATH) lub konieczność uruchomienia wirtualnego środowiska narzędzia Anaconda. 

Interfejs programistyczny biblioteki Tensorflow zapewnił tworzenie skryptów odpowiedzialnych za detekcję przedmiotów znajdujących się na obrazie. Dodatkowo biblioteka ta umożliwia uzyskanie informacji o pozycji lub klasyfikacji obiektu. Dostarczona przez autorów dokumentacja usprawniła naukę sposobu jej działania oraz płynną implementację założeń projektowych. W stworzonym systemie biblioteka Tensorflow została użyta w celu detekcji przedmiotów znajdujących sie na obrazie.
}
